\documentclass[10pt, a4paper]{article}
\usepackage{CJKutf8}
\usepackage{amsmath,amssymb}
\usepackage{mathtools}
\usepackage{hyperref}
\usepackage{multirow}
\usepackage{graphicx}
\usepackage{subfig}
\usepackage{seqsplit}
\graphicspath{ {./img/} }
\setlength{\headsep}{0pt}
\usepackage{geometry}
\geometry{margin=0.9in}
\usepackage{hyperref}


\begin{document}
\begin{CJK}{UTF8}{bkai}
\centering
\huge \textbf{ADLxMLDS2017 - HW4}\\
\raggedleft
\large {電信碩一 宋易霖 r06942076}\linebreak[2]\par
\end{CJK}

\begin{CJK}{UTF8}{bkai}
\section{Problem Define}
利用 conditional adversarial generative network 學出給定的 anime data 的 distribution,並且根據 testing 的條件產生對應的圖片。

\section{Model description}
我的 generator 和 critic 都是參考 DCGAN 的架構,用 convolution stride 取代 pooling layer,並且在 generator 加了 batch normalization,把 critic 的 activation function 換成 leaky relu 等等, 詳細的架構如 Figure \ref{fig:f1}所示。\par

\noindent
我用的 objective funtion 是 WGAN 和 ACGAN 綜合起來的,critic 會有兩種 output,一個是輸出一個 scalar,用來計算真的圖片以及假的圖片的 Wasserstein distance,用途是希望 generator 製作的圖片越真越好, 並且會加上 gradient penalty 使 critic 符合 WGAN 的假設。 另一種是多維度的輸出,每一個維度對應一個頭髮的眼色或是眼睛的顏色 (label),目的是希望 generator 製作出讓 critic 特徵分類正確的圖片。整個對抗性的架構在 Figure \ref{fig:f4}。以下列出 generator 和 critic 兩者的 objective function: \\
\noindent
generator 要 maximize: (f 是 critic 要學習成的function, $P_G$則是 generator 要學習的)
\[ \mathbb{E}_{x \sim P_G}(f(x) + \log P(label | x)) \]
critic 要 maximize:
\[ \mathbb{E}_{x \sim P_{data}}(f(x) + \log P(label | x)) + \mathbb{E}_{x \sim P_G}(-f(x) + \log P(label|x)) - 10\times \mathbb{E}_{x \sim P_{penalty}}(
(||\triangledown_{x}f(x)|| - 1)^2 \]
其中 $logP(label | x) $ 其實就是給定 image 後, critic 輸出與真的 label 的 maximum likelihood,實作上我是 minimize critic 輸出與 label 的 cross entropy。而 $P_{penalty}$ 就如同老師投影片的作法,是從 sample 中真的圖片以及假的圖片隨機比例內插出來的圖片。 
\begin{figure}[!htb]
\centering
\subfloat[Generator (text)]{\includegraphics[scale=0.3]{generator.png}\label{fig:f2}}
\hfill
\subfloat[Critic (text)]{\includegraphics[scale=0.3]{critic.png}\label{fig:f3}}
\hfill
\caption{Model structures}
\label{fig:f1}
\end{figure}

\begin{figure}[!htb]
\centering
\includegraphics[scale=0.5]{GAN_flow.png}

\caption{Generative Learning flow chart}
\label{fig:f4}
\end{figure}


\section{How do you improve your performance }

\subsection{data augmentation}
使用 data augmentation 可以使 critic 看過更多真實的圖片,也加強了 critic 分辨真實照片的能力,如此應該也能夠逼迫 generator 要產生更真實的圖片。
\subsection{increase dimension of noise z}
z 的維度可以看作是 output image 的 feature,所以 z 的維度高低同時也決定了 output image 真正的維度。大多數人在產生 image 的時候都把 z 的維度設成 100 上下,表示他們認為 image 的維度實際上大概就只有100左右。但是我認為如果把維度設高一點,甚至設成跟 output image 同一個維度,這樣照理說可以包含所有 output image 可能的真正維度,這時產生出來的 image 應該也會更真實。 
\subsection{stackGAN}
這部份是聽從老師在社團裡給的建議,讓機器產生 $96 \times 96$ 的圖片後再壓縮成 $64 \times 64$ 的圖片。而要直接產生 $96 \times 96$ 的圖片可能比較困難,因此先用一個 generator 產生 $64 \times 64$ 的圖,再用第二個 generator 產生 $96 \times 96$ 的圖。因為我希望兩個 generator 各司其職,所以不是 end-to-end 的 training,而是分段訓練:先 train 好第一個 generator 後固定這個 generator,再把第一個 generator 的 output 當成第二個 generator 的 input 來 train 第二個 generator。

\section{Experiment settings and observation}
\subsection{Experiment settings}
基本的設定如下 \\
1. batch size: 64 \\
2. z 的分佈: 輸出在 0, 1 之間的 uniform distribution \\
3. optimizer: Adam \\
4. learning rate: 0.0001 \\
5. gradient penalty 參數: 10 \\
6. tags 的取法: 把所有可能的頭髮顏色以及眼睛顏色形成一個 1 of N encoding,若某張圖有符合某個顏色的頭髮或是眼睛,就把該 encoding 對應的維度設成1。\\
7. data augmentation 的方法: sample 每筆data時,隨機左右翻轉 data 或是旋轉 -30 至 30 度。


\subsection{observation}
因為 GAN 不太容易以一個 metric 來衡量 model 的好壞, 以下我對幾種不同的實驗參數或 model 跑出了幾張圖用肉眼來比較他們的好壞。若是沒特別標注的參數就是以 4-1 呈現的參數為準。並且這些圖產生的 noise 是固定 seed 的。\\
A: z: 100, w/o data augmentation (train 500 epochs) \\
B: z: 100, w/ data augmentation (train 500 epochs)\\ 
C: z: 100 w/ data augmentation, stackGAN (兩個 generator 各 train 500 epochs)\\
D: z: 4096 w/ data augmentation (train 2000 epochs)\\

\begin{table}[!htb]
\centering
\begin{tabular}{|c|c|c|c|}
\hline
{\textbf{type of model}} & {\textbf{green hair, blue eyes}} & {\textbf{pink hair, gray eyes}} & \textbf{black 	hair, yellow eyes}\\
\hline
{\textbf{A}} & \raisebox{-\totalheight}{\includegraphics[scale=0.6]{A/sample_0.jpg}}  & \raisebox{-\totalheight}{\includegraphics[scale=0.6]{A/sample_1.jpg}} &  \raisebox{-\totalheight}{\includegraphics[scale=0.6]{A/sample_2.jpg}} \\
\hline
{\textbf{B}} &  \raisebox{-\totalheight}{\includegraphics[scale=0.6]{B/sample_0.jpg}} &  \raisebox{-\totalheight}{\includegraphics[scale=0.6]{B/sample_1.jpg}} &  \raisebox{-\totalheight}{\includegraphics[scale=0.6]{B/sample_2.jpg}} \\
\hline
{\textbf{C}} &  \raisebox{-\totalheight}{\includegraphics[scale=0.6]{C/sample_0.jpg}} &  \raisebox{-\totalheight}{\includegraphics[scale=0.6]{C/sample_1.jpg}} &  \raisebox{-\totalheight}{\includegraphics[scale=0.6]{C/sample_2.jpg}} \\
\hline
{\textbf{D}} &  \raisebox{-\totalheight}{\includegraphics[scale=0.6]{D/sample_0.jpg}} &  \raisebox{-\totalheight}{\includegraphics[scale=0.6]{D/sample_1.jpg}} &  \raisebox{-\totalheight}{\includegraphics[scale=0.6]{D/sample_2.jpg}} \\
\hline

\end{tabular} 
\caption{多種參數設定下 genetator 產生出的圖片}
\label{table:1}
\end{table}
\noindent
從 Table \ref{table:1}可以得到一些觀察:\\
1. A 和 B 的比較可以看出在這個 dataset 做了 augmentation 效果是有變好的,沒有 data augmentation 大部分圖片也都還可以,但是偶爾會出現特別崩壞的圖。\\
2. C 的 model 有兩個 generator,第一個 generator 就是拿 B model 的 generator, 然後固定住這個 generator 再 train 第二個 generator 500 個 epoch。B 和 C 可以比較出:先生成 $96 \times 96$ 再壓縮後的解析度好像有高一點。像是綠頭髮的第二張及第三張都明顯比較清楚,機器也畫得出嘴巴。紅頭髮的狀況下也是比較清楚,特別是眼睛畫的比較好。因此用 stackGAN 的作法的確得到了比較好的結果,但是因為同時也用了比較多參數,所以如果 B 的方法加了參數也不一定會輸 C 這個作法。\\
3. D 的 model train 起來結果是最糟的,圖片不但比較模糊,連 condition 都錯了。不過這樣的結果後來想想也是可以預期的:首先是模糊的問題,因為 z 的 dimension 太大,導致其實在 train 的時候有很多可能性其實 train 不太到,但是 testing 時卻有可能會 sample 到那些沒 train 過的 z,所以結果就很模糊。而 condition 吃不太到的原因也是因為 dimension 有大約4000維,而 condition 的 dimension 大概只有 20 維左右,導致 condition 很容易被稀釋掉,因此產生的圖都對不上 condition。

\section{bonus: style transfer}
\subsection{dataset}
我的 style transfer 做在兩個不同的 dataset 間的轉換:這個作業的 anime dataset 以及 celebA dataset。因為這兩個 dataset 都是把頭像切在中間並且都是人臉,感覺關聯性比較大,可能做起來會比較容易,所以挑了這兩個 dataset。\\
\subsection{implementation}
實做的方法完全按照 cycleGAN,假設 anime dataset 是 domain A, celebA dataset 是 domain B,分別有兩個 critic 以及 generator: generator A 把 domain A 的圖片轉成 domain B, critic A 學會分辨真的 domain A 的圖以及從 domain B 轉過來的假圖,而 generator B 和 critic B 做一樣的事情,只是是應用在 domain B 上。整個 cycleGAN 的架構以及流程如 Figure \ref{fig:f5} 的圖所示。\\
Critic 的架構和 Figure \ref{fig:f3} 一樣, generator 則是從原本的 image 用 convolution layer downsample 三層後,再用 convolution transpose upsample
回原本 image 的大小 (Figure \ref{fig:f6})。 \\
而和一般 GAN 的 objective function 不同的是,cycleGAN 除了原本的 adversarial loss 以外還加上了 cycle loss,以 domain A 的圖片為例: 假設先取一張 domain A 的圖片,先用 generator A 轉成 domain B 的假圖後,再用 generator B 把這假圖轉回 domain A, cycle loss 就是希望轉回來的這張圖和原本的圖 pixel distance 要越近越好,加上了這個 loss 可以讓 generator A 轉換的圖不要和原圖差異太大,不然用 generator B 可能會轉不回原圖,而在 domain B 也是使用一樣的 loss。接著我們可以把generator 和 critic 的 objective function 寫出來 (只寫出一個 domain):\\
generator A 要 minimize adversarial loss 和 cycle loss的和 (D 就是critic 要模擬的 function):
\[\mathbb{E}_{x \sim P_{data, A}}[D_{B}(G_A(x)) - 1]^2 + 10.0 \, \mathcal{L}_{cycle} \]
其中:
\[\mathcal{L}_{cycle} = \mathbb{E}_{x \sim P_{data, A}}[||G_B(G_A(x)) - x||_1] + \mathbb{E}_{y \sim P_{data, B}}[||G_A(G_B(y)) - y||_1]\]
critic A 要 minimize:
\[\mathbb{E}_{x \sim P_{data, A}} [D_A(x) - 1]^2 + \mathbb{E}_{y \sim P_{data, B}}[D_A(G_B(y))^2] \]
需要注意的是這邊 adversarial loss 是要 minimize chi-square distance,而不是 Wasserstein distance,這兩個 distance 我都有做,而使用 WGAN 會讓 generator 太強調 train 成很像另一個 domain 的圖,導致跟原本的圖長比較不像,而用 chi-square distance 的結果是比較好的。
\begin{figure}
\centering
\subfloat[Training flow chart]{\includegraphics[scale=0.45]{style_transfer_flow}\label{fig:f5}}
\hfill
\subfloat[generator structure]{\includegraphics[scale=0.3]{style_generator}\label{fig:f6}}
\hfill
\caption{cycleGAN training flow chart and generator structure}
\label{fig:f7}
\end{figure}

\subsection{result}
Figure \ref{fig:f10} 是我的 style transfer 成果,實驗設定其實跟以上的作法都一樣,這邊的成果大概 train 了 100 epochs,train 太久反而可能會爛掉 (大約 300 的時候) ... 可以發現轉成真人應該是比較困難的,因為有些臉都不太清楚。而動漫人物的頭髮通常誇張的長,所以轉成真人的時候頭髮常常就會轉成背景。反之從真人轉成動畫時,背景就會變成頭髮。\\

\begin{figure}
\centering
\subfloat[Anime samples and their transform]{\includegraphics[scale=0.7]{fake_B}\label{fig:f8}}
\hfill
\subfloat[CelebA samples and their transform]{\includegraphics[scale=0.7]{fake_A}\label{fig:f9}}
\hfill
\caption{Demo of style transfer}
\label{fig:f10}
\end{figure}

\section{Reference}
\noindent

\url{https://arxiv.org/pdf/1703.10593.pdf}\par
\url{https://docs.google.com/presentation/d/1cX5hyAnP5nEsZO5NDfGribA0yfUqGmXt2RV6ETUJhec/edit#slide=id.g2caa8fd3d5_100_0}\par
\url{https://arxiv.org/pdf/1701.07875.pdf}\par

\end{CJK}

\end{document}
